\documentclass{article}
\usepackage[utf8]{inputenc}
\usepackage{graphicx}
\author{G.Adarsh}
\title{Introduction to git}

\begin{document}
\maketitle
Git is a version control system.
\begin{paragraph}
We create a number of versions of a software down the line. Sometimes we need to manipulate the versions based on the revisions we do. It can also be interpreted as, number of people working together differently from the same base and differently. Git provides the above said flexibility. Studying the available commands will give the clear idea of what git is and what it can do.
\end{paragraph}

\begin{section}{Git commands}
All git commands start with "$ git". 
\begin{itemize}
\item $ git --version : It prints the current version of git on the command line as shown in the fig \ref{fig:version}.
\item $ git --help : A simple command of 'git help <command>' will open a detailed description of that particular command. 'git help -a' will list all the available commands in the git. 
\end{itemize}

\begin{itemize}
\item $ git clone <path>: It clones a repository from cloud to local machine. Everyone who has access to clone can clone the git repository(informally, a folder or a directory) into a local machine. The best feature of git is that it is downloaded almost completely onto local machine. therfeore, even in case of loss of data or corrupted, we can easily get back the work that was previously saved/committed.
\item $ git fetch
\end{itemize}
\end{section}
\end{document}
